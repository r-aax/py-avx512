\documentclass[a4paper,12pt]{extarticle}                     % тип документа с размером шрифта 12pt

\usepackage[top=2cm,left=3cm,right=1cm,bottom=2cm]{geometry} % размеры полей
\usepackage[utf8]{inputenc}                                  % эта строка нужна, чтобы документ открывался в редакторе MikTex
\usepackage[T2A]{fontenc}                                    % для поддержки русского языка
\usepackage[russian]{babel}                                  % включение русского языка
\usepackage{amsmath,amsthm,amscd,amsfonts,amssymb}           % специальные символы и т.п.
\usepackage{indentfirst}                                     % отступ для начала абзаца
\usepackage{textcomp}                                        % текст в формулах
\usepackage{graphicx}                                        % подключение графики
\usepackage{caption2}                                        % для изменения стиля подписи рисунков
                                                             % приводит к warning-у, так что использовать только по необходимости

% Использование листингов программного кода.
\usepackage{listings}
\lstset{
language=C++,
basewidth=0.5em,
xleftmargin=45pt,
xrightmargin=45pt,
basicstyle=\small\ttfamily,
keywordstyle=\bfseries\underbar,
numbers=left,
numberstyle=\tiny,
stepnumber=1,
numbersep=10pt,
showspaces=false,
showstringspaces=false,
showtabs=false,
frame=trBL,
tabsize=2,
captionpos=t,
breaklines=true,
breakatwhitespace=false,
escapeinside={\%*}{*)}
}

\renewcommand{\baselinestretch}{1.0}                         % полуторный отступ между строк
\renewcommand{\captionlabeldelim}{.}                         % разделитель между номером рисунка и названием
\numberwithin{equation}{section}                             % нумерация формул по секциям
\numberwithin{figure}{section}                               % нумерация картинок по секциям
\numberwithin{table}{section}                                % нумерация таблиц по секциям

\theoremstyle{plain}                                         % стиль теорем
\newtheorem{theorem}{Теорема}[section]                       % теорема
\newtheorem{lemma}{Лемма}[section]                           % лемма
\newtheorem{definition}{Определение}[section]                % определение

\numberwithin{theorem}{section}                              % нумерация теорем по секциям
\numberwithin{lemma}{section}                                % нумерация лемм по секциям
\numberwithin{definition}{section}                           % нумерация определений по секциям

\begin{document}

\title{Py-AVX512}
\date{2022}
\maketitle
\thispagestyle{empty}                                        % не нумеруем первую страницу

\newpage

\renewcommand{\contentsname}{Оглавление}                     % оглавление. переопределяем команду непосредственно
                                                             % перед генерацией оглавления
\tableofcontents

\newpage

\section*{Введение}                                          % выключить номер введения
\addcontentsline{toc}{section}{Введение}                     % но добавить его в оглавление

TODO

\newpage

\section{Пригодный для векторизации программный контекст}

Будем рассматривать простые функции $f$, которые по набору скалярных данных $a, b, c, \dots$ получают набор других скалярных данных $x, y, z, \dots$.

\begin{equation}
f(a, b, c, \dots) \rightarrow \{ x, y, z, \dots \}
\end{equation}

Теперь вместо скалярных величин $a, b, c, \dots, x, y, z, \dots$ будем рассматривать массивы соответствующих величин одной и той же размерности $A, B, C, \dots, X, Y, Z, \dots$. Для каждого набора величин из этих массивов с одним и тем же индексом $a[i], b[i], c[i], \dots, x[i], y[i], z[i], \dots$ можно рассмотреть применение этой же функции $f$.

\begin{equation}
f(a[i], b[i], c[i], \dots) \rightarrow \{ x[i], y[i], z[i], \dots \}
\end{equation}

Тогда целью данной работы является создание векторного аналога рассматриваемой функции $f$ (обозначим ее через $F$), применение которой можно записать в следующем виде:

\begin{equation}
f(A, B, C, \dots) \rightarrow \{ X, Y, Z, \dots \}
\end{equation}

Таким образом, будем рассматривать элементарную функцию $f$, которая в общем виде объявлена следующим образом:

\begin{lstlisting}[caption={Общий вид скалярной векторизуемой функции.}]
void
f (<32-bit type> *a,
   <32-bit type> *b,
   <32-bit type> *c,
   ...,
   <32-bit type> *x,
   <32-bit type> *y,
   <32-bit type> *z,
   ...)
{
	<read *a, *b, *c, ...>
	
	<calculations>
	
	*x = ...;
	*y = ...;
	*z = ...;    
}
\end{lstlisting}

\

В качестве объекта векторизации исследуется функция, которая обрабатывает массивы входных величин.

\begin{lstlisting}[caption={Общий вид векторизуемой функции, которая содержит вызовы скалярной функции $f$.}]
void
F (<32-bit type> *A,
   <32-bit type> *B,
   <32-bit type> *C,
   ...,
   <32-bit type> *X,
   <32-bit type> *Y,
   <32-bit type> *Z,
   ...,
   int n)
{
	for (int i = 0; i < n; i++)
	{
		f(&a[i], &b[i], &c[i], ...,
		  &x[i], &y[i], &z[i], ...);
	}
}
\end{lstlisting}

\

Без ограничения общности можно считать, что $n$ равно ширине векторизации (количеству скалярных элементов, содержащихся в векторе).

\newpage

\section{Промежуточное представление программы}

\subsection{Поддерживаемые типы данных}

В качестве типов данных, поддерживаемых при выполнении скалярных вычислений, будем рассматривать целочисленные и вещественные значения размером 32 бита. Для управления скалярными вычислениями будем использовать логические предикаты.

Для векторных операций будем рассматривать векторные регистры размером 512 бит (смещающие по 16 32-битных значений). Также для векторных вычислений используются масочные регистры.

\subsection{Объекты данных промежуточного представления}

Ниже перечислены поддержанные объекты данных промежуточного представления:

\begin{itemize}
\item Скалярный регистр, содержащий 32-битное целочисленное значение. Обозначение: \texttt{ri0, ri1, ri2, ...}
\item Скалярный регистр, содержащий 32-битное вещественное значение. Обозначение: \texttt{rf0, rf1, rf2, ...}
\item Предикат, содержащий логическое значение. Обозначение: \texttt{p0, p1, p2, ...}
\item Целочисленная константа. Обозначение: \texttt{ci(0), ci(5), ci(-10), ...}
\item Вещественная константа. Обозначение: \texttt{cf(0.0), cf(3.1415926), cf(-10.0), ...}
\item Векторный регистр, содержащий 16 32-битных целочисленных элементов. Обозначение: \texttt{Ri0, Ri1, Ri2, ...}
\item Векторный регистр, содержащий 16 32-битный вещественных элементов. Обозначение: \texttt{Rf0, Rf1, Rf2, ...}
\item Масочный векторный регистр для использования в векторных командах. Обозначение: \texttt{P0, P1, P2, ...}

\end{itemize}

\subsection{Поддерживаемые инструкции}

Поддерживаемые инструкции разделяются на классы в зависимости от их типов и количества операндов. Для реализуемых скалярных инструкций должны существовать векторные аналоги \cite{IntelSDM}.

\newpage

\subsubsection{Операция с одним операндом и одним результатом-регистром, выполняемая под предикатом}

Название класса операций: \texttt{Arg1\_ResR\_Cnd}

Обозначение операции: \texttt{op(a) -> r ? p}

Операция на вход получает один операнд (для всех операций это просто регистр, но для операции пересылки может быть константа), в качестве результата получает один результат-регистр. Выполнение операции регулируется предикатом (отсутствие предиката эквивалентно истинному предикату). Псевдокод операции имеет следующий вид:

\begin{lstlisting}[caption={Псевдокод операции Arg1\_ResR\_Cnd.}]
if (p)
{
    r = op(a);
}

\end{lstlisting}

\

\begin{table}[!h]
\setcaptionmargin{0mm}
\onelinecaptionsfalse
\captionstyle{flushleft}
\caption{Таблица операций \texttt{Arg1\_ResR\_Cnd}.}
\bigskip
\begin{tabular}{|c|c|c|c|c|c|}
\hline
\textit{Семантика} & \textit{Арг-ты} & \textit{Рез.} & \textit{Скал. оп-ция} & \textit{Вект. оп-ции} & \textit{Инстр. AVX-512} \\
\hline
\texttt{=}           & \texttt{f}  & \texttt{f} & \texttt{movf}  & \texttt{MOVf}  & \texttt{MOVUPS} \\
\texttt{=}           & \texttt{cf} & \texttt{f} & \texttt{movf}  & \texttt{MOVf}  & \texttt{-} \\
\texttt{=}           & \texttt{i}  & \texttt{i} & \texttt{movi}  & \texttt{MOVi}  & \texttt{-} \\
\texttt{=}           & \texttt{ci} & \texttt{i} & \texttt{movi}  & \texttt{MOVi}  & \texttt{-} \\
\texttt{abs(a)}      & \texttt{f}  & \texttt{f} & \texttt{absf}  & \texttt{ABSf}  & \texttt{VANDPS} \\
\texttt{abs(a)}      & \texttt{i}  & \texttt{i} & \texttt{absi}  & \texttt{ABSi}  & \texttt{VPABSD} \\
\texttt{sqrt(a)}     & \texttt{f}  & \texttt{f} & \texttt{sqrtf} & \texttt{SQRTf} & \texttt{VSQRTPS} \\
\texttt{1.0 / a}     & \texttt{f}  & \texttt{f} & \texttt{rcpf}  & \texttt{RCPf}  & \texttt{VRCP14PS} \\
\texttt{(int32)a}    & \texttt{f}  & \texttt{i} & \texttt{f2i}   & \texttt{F2I}   & \texttt{VCVTPS2DQ} \\
\texttt{(float32)a}  & \texttt{i}  & \texttt{f} & \texttt{i2f}   & \texttt{I2F}   & \texttt{VCVTDQ2PS} \\
\texttt{2.0 \^ \ a } & \texttt{f}  & \texttt{f} & \texttt{exp2f} & \texttt{EXP2f} & \texttt{VEXP2PS} \\
\hline
\end{tabular}
\end{table}

\newpage

\subsubsection{Операция с двумя операндами-регистрами и одним результатом-регистром, выполняемая под предикатом}

Название класса операций: \texttt{ArgRR\_ResR\_Cnd}

Обозначение операции: \texttt{op(a, b) -> r ? p}

Операция на вход получает два операнда-регистра, в качестве результата получает один результат-регистр. Выполнение операции регулируется предикатом (отсутствие предиката эквивалентно истинному предикату). Псевдокод операции имеет следующий вид:

\begin{lstlisting}[caption={Псевдокод операции ArgRR\_ResR\_Cnd.}]
if (p)
{
    r = op(a, b);
}

\end{lstlisting}

\

\begin{table}[!h]
\setcaptionmargin{0mm}
\onelinecaptionsfalse
\captionstyle{flushleft}
\caption{Таблица операций \texttt{ArgRR\_ResR\_Cnd}.}
\bigskip
\begin{tabular}{|c|c|c|c|c|c|}
\hline
\textit{Семантика} & \textit{Арг-ты} & \textit{Рез.} & \textit{Скал. оп-ция} & \textit{Вект. оп-ции} & \textit{Инстр. AVX-512} \\
\hline
\texttt{a + b}      & \texttt{ff} & \texttt{f} & \texttt{addf}  & \texttt{ADDf}  & \texttt{VADDPS} \\
\texttt{a + b}      & \texttt{ii} & \texttt{i} & \texttt{addi}  & \texttt{ADDi}  & \texttt{VPADDD} \\
\texttt{a - b}      & \texttt{ff} & \texttt{f} & \texttt{subf}  & \texttt{SUBf}  & \texttt{VSUBPS} \\
\texttt{a - b}      & \texttt{ii} & \texttt{i} & \texttt{subi}  & \texttt{SUBi}  & \texttt{VPSUBD} \\
\texttt{a * b}      & \texttt{ff} & \texttt{f} & \texttt{mulf}  & \texttt{MULf}  & \texttt{VMULPS} \\
\texttt{a * b}      & \texttt{ii} & \texttt{i} & \texttt{muli}  & \texttt{MULi}  & \texttt{VPMULLD} \\
\texttt{a / b}      & \texttt{ff} & \texttt{f} & \texttt{divf}  & \texttt{DIVf}  & \texttt{VDIVPS} \\
\texttt{a \& b}     & \texttt{ii} & \texttt{i} & \texttt{andi}  & \texttt{ANDi}  & \texttt{VPANDD} \\
\texttt{a \& \~{}b} & \texttt{ii} & \texttt{i} & \texttt{andni} & \texttt{ANDNi} & \texttt{VPANDND} \\
\texttt{a | b}      & \texttt{ii} & \texttt{i} & \texttt{ori}   & \texttt{ORi}   & \texttt{VPOR} \\
\texttt{max(a, b)}  & \texttt{ff} & \texttt{f} & \texttt{maxf}  & \texttt{MAXf}  & \texttt{VMAXPS} \\
\texttt{max(a, b)}  & \texttt{ii} & \texttt{i} & \texttt{maxi}  & \texttt{MAXi}  & \texttt{VPMAXSD} \\
\texttt{min(a, b)}  & \texttt{ff} & \texttt{f} & \texttt{minf}  & \texttt{MINf}  & \texttt{VMINPS} \\
\texttt{min(a, b)}  & \texttt{ii} & \texttt{i} & \texttt{mini}  & \texttt{MINi}  & \texttt{VPMINSD} \\
\texttt{pow(a, b}   & \texttt{ff} & \texttt{f} & \texttt{powf}  & \texttt{POWi}  & \texttt{-} \\
\hline
\end{tabular}
\end{table}

\newpage

\subsubsection{Операция с тремя операндами-регистрами и одним результатом-регистром, выполняемая под предикатом}

Название класса операций: \texttt{ArgRRR\_ResR\_Cnd}

Обозначение операции: \texttt{op(a, b, c) -> r ? p}

Операция на вход получает три операнда-регистра, в качестве результата получает один результат-регистр. Выполнение операции регулируется предикатом (отсутствие предиката эквивалентно истинному предикату). Псевдокод операции имеет следующий вид:

\begin{lstlisting}[caption={Псевдокод операции ArgRRR\_ResR\_Cnd.}]
if (p)
{
    r = op(a, b, c);
}

\end{lstlisting}

\

\begin{table}[!h]
\setcaptionmargin{0mm}
\onelinecaptionsfalse
\captionstyle{flushleft}
\caption{Таблица операций \texttt{ArgRRR\_ResR\_Cnd}.}
\bigskip
\begin{tabular}{|c|c|c|c|c|c|}
\hline
\textit{Семантика} & \textit{Арг-ты} & \textit{Рез.} & \textit{Скал. оп-ция} & \textit{Вект. оп-ции} & \textit{Инстр. AVX-512} \\
\hline
\texttt{a * b + c}  & \texttt{fff} & \texttt{f} & \texttt{fmaddf}  & \texttt{FMADDf}  & \texttt{VFMADD*PS} \\
\texttt{a * b - c}  & \texttt{fff} & \texttt{f} & \texttt{fmsubf}  & \texttt{FMSUBf}  & \texttt{VFMSUB*PS} \\
\texttt{-a * b + c} & \texttt{fff} & \texttt{f} & \texttt{fnmaddf} & \texttt{FNMADDf} & \texttt{VFNMADD*PS} \\
\texttt{-a * b - c} & \texttt{fff} & \texttt{f} & \texttt{fnmsubf} & \texttt{FNMSUBf} & \texttt{VFNMSUB*PS} \\
\hline
\end{tabular}
\end{table}

\newpage

\subsubsection{Операция с двумя операндами-регистрами и одним результатом-предикатом, выполняемая под предикатом}

Название класса операций: \texttt{ArgRR\_ResP\_Cnd}

Обозначение операции: \texttt{op(a, b) -> q ? p}

Операция на вход получает два операнда-регистра, в качестве результата получает один результат-предикат. Выполнение операции регулируется предикатом (отсутствие предиката эквивалентно истинному предикату). Псевдокод операции имеет следующий вид:

\begin{lstlisting}[caption={Псевдокод операции ArgRR\_ResP\_Cnd.}]
if (p)
{
    q = op(a, b);
}

\end{lstlisting}

\

\begin{table}[!h]
\setcaptionmargin{0mm}
\onelinecaptionsfalse
\captionstyle{flushleft}
\caption{Таблица операций \texttt{ArgRR\_ResP\_Cnd}.}
\bigskip
\begin{tabular}{|c|c|c|c|c|c|}
\hline
\textit{Семантика} & \textit{Арг-ты} & \textit{Рез.} & \textit{Скал. оп-ция} & \textit{Вект. оп-ции} & \textit{Инстр. AVX-512} \\
\hline
\texttt{a == b} & \texttt{ff} & \texttt{p} & \texttt{cmpeqf} & \texttt{CMPEQf} & \texttt{VCMPPS} \\
\texttt{a == b} & \texttt{ii} & \texttt{p} & \texttt{cmpeqi} & \texttt{CMPEQi} & \texttt{VCMPPD} \\
\texttt{a != b} & \texttt{ff} & \texttt{p} & \texttt{cmpnef} & \texttt{CMPNEf} & \texttt{VCMPPS} \\
\texttt{a != b} & \texttt{ii} & \texttt{p} & \texttt{cmpnei} & \texttt{CMPNEi} & \texttt{VCMPPD} \\
\texttt{a < b}  & \texttt{ff} & \texttt{p} & \texttt{cmpltf} & \texttt{CMPLTf} & \texttt{VCMPPS} \\
\texttt{a < b}  & \texttt{ii} & \texttt{p} & \texttt{cmplti} & \texttt{CMPLTi} & \texttt{VCMPPD} \\
\texttt{a <= b} & \texttt{ff} & \texttt{p} & \texttt{cmplef} & \texttt{CMPLEf} & \texttt{VCMPPS} \\
\texttt{a <= b} & \texttt{ii} & \texttt{p} & \texttt{cmplei} & \texttt{CMPLEi} & \texttt{VCMPPD} \\
\texttt{a > b}  & \texttt{ff} & \texttt{p} & \texttt{cmpgtf} & \texttt{CMPGTf} & \texttt{VCMPPS} \\
\texttt{a > b}  & \texttt{ii} & \texttt{p} & \texttt{cmpgti} & \texttt{CMPGTi} & \texttt{VCMPPD} \\
\texttt{a >= b} & \texttt{ff} & \texttt{p} & \texttt{cmpgef} & \texttt{CMPGEf} & \texttt{VCMPPS} \\
\texttt{a >= b} & \texttt{ii} & \texttt{p} & \texttt{cmpgei} & \texttt{CMPGEi} & \texttt{VCMPPD} \\
\hline
\end{tabular}
\end{table}

\newpage

\subsubsection{Операция выбора одного из двух значений по предикату}

Название класса операций: \texttt{Blend}

Обозначение операции: \texttt{blend a, b, q -> r}

Данный класс содержит всего одну операцию. Операция на вход получает два операнда-регистра и предикат, в качестве результата получает один результат-регистр. Псевдокод операции имеет следующий вид:

\begin{lstlisting}[caption={Псевдокод операции Blend.}]
if (!p)
{
    r = a;
}
else
{
    r = b;
}

\end{lstlisting}

\

\begin{table}[!h]
\setcaptionmargin{0mm}
\onelinecaptionsfalse
\captionstyle{flushleft}
\caption{Таблица операций \texttt{Blend}.}
\bigskip
\begin{tabular}{|c|c|c|c|c|c|}
\hline
\textit{Семантика} & \textit{Арг-ты} & \textit{Рез.} & \textit{Скал. оп-ция} & \textit{Вект. оп-ции} & \textit{Инстр. AVX-512} \\
\hline
\texttt{!p ? a : b} & \texttt{ff} & \texttt{f} & \texttt{blendf} & \texttt{BLENDf} & \texttt{VBLENDPS} \\
\texttt{!p ? a : b} & \texttt{ii} & \texttt{i} & \texttt{blendi} & \texttt{BLENDi} & \texttt{-} \\
\hline
\end{tabular}
\end{table}

\newpage

\subsubsection{Операция с одним операндом-предикатом и одним результатом-предикатом}

Название класса операций: \texttt{ArgP\_ResP}

Обозначение операции: \texttt{op(p) -> q}

Операция на вход получает один операнд-предикат, в качестве результата получает один результат-предикат. Псевдокод операции имеет следующий вид:

\begin{lstlisting}[caption={Псевдокод операции ArgP\_ResP.}]
q = op(p);
\end{lstlisting}

\

\begin{table}[!h]
\setcaptionmargin{0mm}
\onelinecaptionsfalse
\captionstyle{flushleft}
\caption{Таблица операций \texttt{ArgP\_ResP}.}
\bigskip
\begin{tabular}{|c|c|c|c|c|c|}
\hline
\textit{Семантика} & \textit{Арг-ты} & \textit{Рез.} & \textit{Скал. оп-ция} & \textit{Вект. оп-ции} & \textit{Инстр. AVX-512} \\
\hline
\texttt{=} & \texttt{p} & \texttt{p} & \texttt{kmov} & \texttt{KMOV} & \texttt{KMOVW} \\
\texttt{!} & \texttt{p} & \texttt{p} & \texttt{knot} & \texttt{KNOT} & \texttt{KNOTW} \\
\hline
\end{tabular}
\end{table}

\newpage

\subsubsection{Операция с двумя операндами-предикатами и одним результатом-предикатом}

Название класса операций: \texttt{ArgPP\_ResP}

Обозначение операции: \texttt{op(p, s) -> q}

Операция на вход получает два операнда-предиката, в качестве результата получает один результат-предикат. Псевдокод операции имеет следующий вид:

\begin{lstlisting}[caption={Псевдокод операции ArgPP\_ResP.}]
q = op(p, s);
\end{lstlisting}

\

\begin{table}[!h]
\setcaptionmargin{0mm}
\onelinecaptionsfalse
\captionstyle{flushleft}
\caption{Таблица операций \texttt{ArgPP\_ResP}.}
\bigskip
\begin{tabular}{|c|c|c|c|c|c|}
\hline
\textit{Семантика} & \textit{Арг-ты} & \textit{Рез.} & \textit{Скал. оп-ция} & \textit{Вект. оп-ции} & \textit{Инстр. AVX-512} \\
\hline
\texttt{p \& s}      & \texttt{pp} & \texttt{p} & \texttt{kand}   & \texttt{KAND}   & \texttt{KANDW} \\
\texttt{p \& !s}     & \texttt{pp} & \texttt{p} & \texttt{kandn}  & \texttt{KANDN}  & \texttt{KANDNW} \\
\texttt{p | s}       & \texttt{pp} & \texttt{p} & \texttt{kor}    & \texttt{KOR}    & \texttt{KORW} \\
\texttt{p \^ \ s}    & \texttt{pp} & \texttt{p} & \texttt{kxor}   & \texttt{KXOR}   & \texttt{KXORW} \\
\texttt{!(p \^ \ s)} & \texttt{pp} & \texttt{p} & \texttt{kxnor}  & \texttt{KXNOR}  & \texttt{KXNORW} \\
\hline
\end{tabular}
\end{table}

\newpage

\subsubsection{Операция чтения из памяти}

Название класса операций: \texttt{Load\_Cnd}

Обозначение операции: \texttt{ld addr -> r ? p}

Данный класс содержит операции чтения. Операция на вход получает адрес и предикат, в качестве результата получает один результат-регистр. Псевдокод операции имеет следующий вид:

\begin{lstlisting}[caption={Псевдокод операции Load\_Cnd.}]
if (p)
{
    r = *addr;
}

\end{lstlisting}

\

\begin{table}[!h]
\setcaptionmargin{0mm}
\onelinecaptionsfalse
\captionstyle{flushleft}
\caption{Таблица операций \texttt{Load\_Cnd}.}
\bigskip
\begin{tabular}{|c|c|c|c|c|c|}
\hline
\textit{Семантика} & \textit{Арг-ты} & \textit{Рез.} & \textit{Скал. оп-ция} & \textit{Вект. оп-ции} & \textit{Инстр. AVX-512} \\
\hline
\texttt{ld(addr)} & \texttt{addr} & \texttt{f} & \texttt{ldf} & \texttt{LDf} & \texttt{MOVUPS} \\
\hline
\end{tabular}
\end{table}

\newpage

\subsubsection{Операция записи в память}

Название класса операций: \texttt{Store\_Cnd}

Обозначение операции: \texttt{st addr, a ? p}

Данный класс содержит операции записи в память. Операция на вход получает адрес, значение для записи и предикат, результат у операции отсутствует. Псевдокод операции имеет следующий вид:

\begin{lstlisting}[caption={Псевдокод операции Store\_Cnd.}]
if (p)
{
    *addr = a;
}

\end{lstlisting}

\

\begin{table}[!h]
\setcaptionmargin{0mm}
\onelinecaptionsfalse
\captionstyle{flushleft}
\caption{Таблица операций \texttt{Store\_Cnd}.}
\bigskip
\begin{tabular}{|c|c|c|c|c|c|}
\hline
\textit{Семантика} & \textit{Арг-ты} & \textit{Рез.} & \textit{Скал. оп-ция} & \textit{Вект. оп-ции} & \textit{Инстр. AVX-512} \\
\hline
\texttt{st(addr)} & \texttt{addr,f} & \texttt{-} & \texttt{stf} & \texttt{STf} & \texttt{MOVUPS} \\
\hline
\end{tabular}
\end{table}

\newpage

\subsubsection{Операция перехода}

Название класса операций: \texttt{Jump\_Cnd}

Обозначение операции: \texttt{jump n ? p}

Данный класс содержит одну операцию - перехода по условию. Операция получает в качестве параметра номер линейного участка промежуточного представления. Если предикат отсутствует, то он считает истинным. Команда имеет только скалярную семантику. Семантика операции выглядит следующим образом:

\begin{lstlisting}[caption={Псевдокод операции Jump\_Cnd.}]
if (p)
{
	jump <block n>;
}

\end{lstlisting}

\

\begin{table}[!h]
\setcaptionmargin{0mm}
\onelinecaptionsfalse
\captionstyle{flushleft}
\caption{Таблица операций \texttt{Jump\_Cnd}.}
\bigskip
\begin{tabular}{|c|c|c|c|c|c|}
\hline
\textit{Семантика} & \textit{Арг-ты} & \textit{Рез.} & \textit{Скал. оп-ция} & \textit{Вект. оп-ции} & \textit{Инстр. AVX-512} \\
\hline
\texttt{jump n} & \texttt{i} & \texttt{-} & \texttt{jump} & \texttt{-} & \texttt{-} \\
\hline
\end{tabular}
\end{table}

\newpage

\section*{Заключение}                                        % выключить номер заключения
\addcontentsline{toc}{section}{Заключение}                   % но добавить его в оглавление

TODO

\newpage

% Список используемой литературы. 
\input bibliography.tex

\end{document}
